\documentclass{article}
\input{header}


\title{Problem Set 2 (Main classes 5-7)}
\author{Izzy Grosof}
\date{October 3, 2025 -- Due October 10th (Friday)}
\begin{document}
\maketitle

\begin{enumerate}
\item Let $X_1, X_2$ be i.i.d. continuous random variables,
both having CDF $F_X(x)$ and density $f_X(x)$.

Let $M = \min(X_1, X_2)$ be the minimum of the two random variables.
\begin{enumerate}
    \item Prove that the CDF of $M$ is:
    \begin{align*}
        F_M(x) = 1 - (1-F_X(x))^2
    \end{align*}
    \item What is the density $f_M(x)$ of $M$ in terms of the density $f_X(x)$ and CDF $F_X(x)$ of $X_1$ and $X_2$?
    \item Suppose that $X_1, X_2 \sim Exp(\lambda)$.
    What is the CDF $F_M(x)$ of $M$?

    \textbf{Note:} Info on the exponential random variable is available in IPC 7.2.
    \item Is $M$ an exponential random variable? Why?
\end{enumerate}
\item You've just shown upon to a train platform,
where trains arrive every 15 minutes.
When you arrive let $T$ denote the time single the last train.
In this case, $T \sim Unif(0, 15)$ minutes.

Let $S$ denote the number of people who you see at the train platform, when you arrive.
People have been arriving to the train platform single the last train arrived at a rate of 2 people per minute, on average.
As a result, $S \sim Poisson(2 T)$.

\textbf{Note:} Info on the Poisson distribution can be found in IPC 3.2.4. Info on the Uniform distribution can be found in IPC 7.2. Note also that $\E{Poisson(\lambda)} = \lambda$ and $\Var{Poisson(\lambda)} = \lambda$.
\begin{enumerate}
    \item What is the mean number of people you see at the train platform, $\E{S}$?
    \item What is the variance of the number of people you see at the train platform, $\Var{S}$?
\end{enumerate}

\item You've been selected as a contestant on Jeopardy, a game show.
You'll play in the first round, where you might win or lose.
Each round, if you win, you go on to the next round, and play again.
Otherwise, you're done.

You're pretty good at Jeopardy, so you have an $80\%$ probability of winning each round, independent of all other rounds.

Let $R$ denote the total number of rounds you play.

\begin{enumerate}
    \item What's the standard name for the distribution of the random variable $R$?

    \textbf{Note:} Standard names for discrete distributions can be found in the textbook, in chapter 3.2.
    \item You've won your first 5 rounds. Let $Q$ denote the number of rounds you will play after these first five. In terms of $R$, we can write $Q$ as follows:
    \begin{align*}
        Q = [R-5 \mid R > 5]
    \end{align*}
    What is the PMF of $Q$, $\P{Q = i}$ for each possible value $i$?
    
    \item What's the standard name for the distribution of the random variable $Q$?
\end{enumerate}

\item Isabella is playing chess online, and keeping track of her rating.
Her initial rating is 1000. Each day, her rating changes by an amount $D_i$, where $D_i \sim Uniform(-30, 30)$ is uniformly distributed between -30 and +30 rating. The rating changes $D_i$ are i.i.d.

After 100 days, approximate Isabella's new rating with a Normal distribution. What's the approximate probability that her new rating is at least 1200?

\item This is a programming problem. Write a program in Python, as either a standalone file or as a Jupyter notebook.
    Include your answers in your solution directly, and also submit the code you write, either in the same document or as a separate upload.

    Simulate the following situation:
    
    Moss is flipping a fair coin (equally likely heads or tails, i.i.d.), until the result is heads.
    
    For each coin flip, Moss waits $Exp(1)$ seconds before flipping the coin. The first coin flip happens at time $Exp(1)$,
    the second coin flip, if there is one, happens $Exp(1)$ time later, and so on.

    Let $T$ denote the time when the final coin, the one whose result is heads, is flipped.

    Generate $10^6$ samples of the random variable $T$.
    \begin{enumerate}
    \item What are $\P{T \ge 1}, \P{T \ge 2}, \P{T \ge 3}, \P{T \ge 4}$?
    \item Moss thinks that $T$ is exponentially distributed,
    and wants to use your data to see if this might be true.
    If a random variable $X \sim Exp(\lambda)$
    is exponentially distributed with rate $\lambda$,
    then for any threshold $x$, the following fraction will be constant:
    \begin{align*}
        \frac{\P{X \ge x+1}}{\P{X \ge x}} = e^{-\lambda}
    \end{align*}
    If $X$ is not exponentially distributed, then the above fraction won't have a constant value across different thresholds $x$.

    Use your data from (a) to check whether the above fraction is near-constant for the random variable $T$ across thresholds $x = \{1, 2, 3\}$.
    
    Does $T$ seem exponentially distributed? If so, what is its rate parameter $\lambda$?
    \end{enumerate}
\end{enumerate}
\end{document}