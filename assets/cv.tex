% LaTeX file for resume
% This file uses the resume document class (res.cls)

\documentclass{res}
%\usepackage{helvetica} % uses helvetica postscript font (download helvetica.sty)
%\usepackage{newcent}   % uses new century schoolbook postscript font
\newsectionwidth{0pt}  % So the text is not indented under section headings
\usepackage{fancyhdr}  % use this package to get a 2 line header
\renewcommand{\headrulewidth}{0pt} % suppress line drawn by default by fancyhdr
\setlength{\headheight}{24pt} % allow room for 2-line header
\setlength{\headsep}{24pt}  % space between header and text
\setlength{\headheight}{24pt} % allow room for 2-line header
\pagestyle{fancy}     % set pagestyle for document
\rhead{ {\it Isaac Grosof}\\{\it p. \thepage} } % put text in header (right side)
\cfoot{}                                     % the foot is empty
\usepackage[margin=0.5in]{geometry}
\usepackage{hyperref}

\begin{document}
\thispagestyle{empty} % this page has no header
\name{\Large ISAAC GROSOF\\[12pt]}% the \\[12pt] adds a blank line after name

\address{Computer Science Department \\
Carnegie Mellon University \\
  Pittsburgh, PA, 15213}
  \address{
      isaacg1.github.io \\
igrosof@cmu.edu \\
  (206) 718-6712}

\begin{resume}

\section{\centerline{RESEARCH INTERESTS}}
    \vspace{4pt}
Design and performance analysis of stochastic computer systems, including theory and implementation. \\
Queueing behavior and scheduling policies for resource allocation, especially in multicore systems.

\section{\centerline{CURRENT RESEARCH}}
    \vspace{4pt}
Multiserver scheduling policies, scheduling policies for tail metrics.

\section{\centerline{EDUCATION}}
    \vspace{4pt}
\textbf{2017 - pres}: Pursuing PhD in Computer Science. Carnegie Mellon University, Pittsburgh, PA.\\
Advisor: Prof. Mor Harchol-Balter

\textbf{2013 - 2017}: M.E. and B.S. in Computer Science. Massachusetts Institute of Technology, Cambridge, MA. GPA 4.96/5 \\
Master's Thesis in information-theoretic cryptography: ``Secure communication: CDS, PIR, PSM'' \\
Master's Advisor: Prof. Vinod Vaikunatanathan \\
Bachelor's Advisor: Prof. Frans Kaashoek

\section{\centerline{REFEREED PUBLICATIONS}}
    \vspace{4pt}

    Ziv Scully, Isaac Grosof, Mor Harchol-Balter.
    ``Optimal Multiserver Scheduling with Unknown Job Sizes in Heavy Traffic.''
    \textit{38th International Symposium on Computer Performance, Modeling, Measurements, and Evaluation (Performance 2020)},
    Milan, Italy, November 2020. To appear.

	Ben Berg, Daniel Berger, Sara McAllister, Isaac Grosof, Sathya Gunasekar, Jimmy Lu, Michael Uhlar,
	Jim Carrig, Nathan Beckmann, Mor Harchol-Balter, Greg Ganger.
	``The CacheLib Caching Engine: Design and Experiences at Scale."
	14th USENIX Symposium on Operating Systems Design and Implementation (OSDI 2020),
	Banff, Canada, November 2020. To appear.

    Isaac Grosof, Ziv Scully, Mor Harchol-Balter.
    ``Load Balancing Guardrails: Keeping Your Heavy Traffic on the Road to Low Response Times.''
    \textit{Proceedings of the ACM Measurement and Analysis of Computer Systems -- SIGMETRICS}:
    Volume 3, Issue 2, Article 42 (June 2019), pp. 42:1 -- 42:31, 2019.

    Conference version appeared in \textit{Proceedings of ACM Simetrics/Performance 2019
    Conference on Measurement and Modeling of Computer Systems (SIGMETRICS 19)}, Pheonix, AZ.
    June 2019.\\
    \textbf{Winner of SIGMETRICS 2019 Best Student Paper Award}.

    Isaac Grosof, Ziv Sculy, Mor Harchol-Balter.
    ``SRPT for Multiserver Systems.''
    \textit{Performance Evaluation}, vol. 127-128, Nov. 2018, pp. 154-175.

    Conference version appeared in \textit{36th International Symposium on Computer Performance, Modeling, Measurements, and Evaluation (Performance 2018)}, Toulouse, France, December 2018. \\
    \textbf{Winner of Performance 2018 Best Student Paper Award}.

    Erik D. Demaine, Isaac Grosof, Jayson Lynch, and Mikhail Rudoy.
    ``Computational Complexity of Motion Planning of a Robot through Simple Gadgets.''
    \textit{Ninth International Conference on Fun with Algorithms}. La Maddalena, Italy. 2018.

    Erik D. Demaine, Isaac Grosof, and Jayson Lynch.
    ``Push-Pull Block Puzzles are Hard.''
    \textit{International Conference on Algorithms and Complexity}. Athens, Greece. 2017.

    Benjamin Grosof, Janine Bloomfield, Paul Fodor, Michael Kifer, Isaac Grosof, Miguel Calejo, and Theresa Swift.
    ``Automated Decision Support for Financial Regulatory/Policy Compliance, using Textual Rulelog.'' \textit{RuleML 2015}. Berlin, Germany. 2015.

    \section{\centerline{OTHER PUBLICATIONS}}
    \vspace{4pt}

    Isaac Grosof. ``Open Problem - M/G/k/SRPT Under Medium Load.''
    \textit{Stochastic Systems}. Sep. 2019.
    
    
\section{\centerline{EMPLOYMENT}}
    \vspace{4pt}
\textbf{Summer 2019}: Facebook, Menlo Park, CA. \\
- Research Intern to develop a machine-learning-based SSD admission policy for Facebook's
Tao caching architecture.\\
- Improved the tradeoff between hit ratio and SSD write rate.

    \textbf{Summer 2018}: Microsoft Research, Seattle, WA. \\
    - Research Intern to develop novel FPGA algorithms for linear algebra
    and solving linear programs.

\textbf{Summer 2016}: Jane Street Capital, LLC, New York City, NY.\\
- Software developer for non-obtrusive data collection about in-house trading.\\
- Software developer responsible for updating trading simulation package to accommodate new trade specification format.

\textbf{Summer 2015}: Coherent Knowledge, Seattle, WA. \\
- Knowledge Engineer to build demonstrations for the financial and natural language domains using the declarative logic programming language Ergo.

\textbf{2013 - 2014}: MIT Undergraduate Research Opportunities Program, Cambridge, MA. \\
- Researcher in Complexity Theory proving computational hardness of block puzzles and related agent motion problems.

\textbf{Summer 2014}: EMC Isilon, Seattle, WA. \\
- Software developer to replace the previous ad-hoc build platform with a modern Jenkins-based build platform.

%% Mention Stack exchange, one of the most used?
%% Mention lines of code - over 20,000 lines of code.

 \section{\centerline{PROJECTS}}
    \vspace{4pt}
 \textbf{2014 - pres}: Author of new programming language: \textit{Pyth}\\
 - Pyth is one of the best programming languages for solving tasks with the shortest possible programs.\\
 - Pyth is an open-source language written in Python which has an online interpreter and detailed documentation.\\
 - Available at \url{https://github.com/isaacg1/pyth}

\section{\centerline{ SKILLS }}
\vspace{4pt}
Strong mathematical background, including abstract logic, probability, and mathematical proofs. \\
Proficient with Python, OCaml, Rust, C++. \\
Experience with Haskell, C, Java.\\
Experience in functional programming (e.g., Scheme) and logic programming (e.g., Prolog). \\
Experience with Linux, Linux shell scripting. \\

\end{resume}
\end{document}
