% LaTeX file for resume
% This file uses the resume document class (res.cls)

\documentclass{res}
%\usepackage{helvetica} % uses helvetica postscript font (download helvetica.sty)
%\usepackage{newcent}   % uses new century schoolbook postscript font
\newsectionwidth{0pt}  % So the text is not indented under section headings
\usepackage{fancyhdr}  % use this package to get a 2 line header
\renewcommand{\headrulewidth}{0pt} % suppress line drawn by default by fancyhdr
\setlength{\headheight}{24pt} % allow room for 2-line header
\setlength{\headsep}{24pt}  % space between header and text
\setlength{\headheight}{24pt} % allow room for 2-line header
\pagestyle{fancy}     % set pagestyle for document
\rhead{ {\it Isaac Grosof}\\{\it p. \thepage} } % put text in header (right side)
\cfoot{}                                     % the foot is empty
\usepackage[margin=0.5in]{geometry}
\usepackage{hyperref}

\begin{document}
\thispagestyle{empty} % this page has no header
\name{\Large ISAAC GROSOF\\[12pt]}% the \\[12pt] adds a blank line after name

\address{School of Industrial and Systems Engineering \\
Georgia Institute of Technology \\
Atlanta, GA}
  \address{
      My website: \href{https://isaacg1.github.io}{isaacg1.github.io} \\
igrosof3@gatech.edu \\
  (206) 718-6712}

\begin{resume}

\section{\centerline{RESEARCH INTERESTS}}
    \vspace{4pt}
Optimization and performance analysis of stochastic scheduling models for modern computing systems,
especially multiserver systems, multiserver jobs, tail latency, scheduling with predictions, and reinforcement learning for queueing.

\section{\centerline{ACADEMIC EMPLOYMENT}}
    \vspace{4pt}

\textbf{9/2024 -}: Assistant Professor (Tenure-track) at Northwestern University,
Department of Industrial Engineering and Management Science.

\textbf{2/2024 - 8/2024}: Postdoctoral Fellow at University of Illinois at Urbana-Champaign,
advised by Prof.~R.~Srikant \& Prof.~Siva Theja Maguluri.

\textbf{8/2023 - present}: Tennenbaum Postdoctoral Fellow at Georgia Tech,
advised by Prof.~Siva Theja Maguluri \& Prof.~R.~Srikant.

\section{\centerline{EDUCATION}}
    \vspace{4pt}
\textbf{2017 - 2023}: PhD in Computer Science. Carnegie Mellon University, Pittsburgh, PA.\\
Thesis title: ``Optimal Scheduling in Multiserver Queues''.
Advisor: Prof.~Mor Harchol-Balter. \\
\textbf{Winner of ACM SIGMETRICS 2023 Doctoral Dissertation Award} 

\textbf{2013 - 2017}: M.E. and B.S. in Computer Science. Massachusetts Institute of Technology, Cambridge, MA. GPA 4.96/5 \\
Master's Thesis in information-theoretic cryptography: ``Secure communication: CDS, PIR, PSM'' \\
Master's Advisor: Prof.~Vinod Vaikunatanathan \\
Bachelor's Advisor: Prof.~Frans Kaashoek

\section{\centerline{HONORS}}
    \vspace{4pt}

ACM SIGMETRICS 2023 Doctoral Dissertation Award \\
INFORMS 2022 George Nicholson Award \\
2023 Siebel Scholar \\
ACM SIGMETRICS 2021 Best Paper Award \\
ACM SIGMETRICS 2019 Best Student Paper Award \\
IFIP Performance 2018 Best Student Paper Award \\
ACM SIGMETRICS 2023 Student Research Competition Winner

\section{\centerline{ADVISING}}

\textbf{Ziyuan Wang}: Began advising Spring 2024. Likely to propose Spring 2025.

\textbf{Cameron Curtis}: Began advising Winter 2024.

\section{\centerline{REFEREED PUBLICATIONS}}
\vspace{16pt}
    \centerline{\href{https://isaacg1.github.io/publications/}{Link to my publications}}

    Runhan Xie, Isaac Grosof, Ziv Scully.
    ``Heavy-Traffic Optimal Size- and State-Aware Dispatching.''
    \textit{ACM SIGMETRICS / IFIP Performance}, Venice, Italy, June 2024.


    Isaac Grosof, Yige Hong, Mor Harchol-Balter, Alan Scheller-Wolf.
    ``The RESET and MARC Techniques, with Application to Multiserver-Job Analysis.''
    \textit{IFIP Performance}, Chicago, IL, November 2023. \\
    A short version of this paper, entitled ``The RESET Technique for Multiserver-Job Analysis,''
    was the \textbf{winner of the ACM SIGMETRICS 2023 Student Research Competition}.

    Isaac Grosof, Ziv Scully, Mor Harchol-Balter, Alan Scheller-Wolf.
    ``Optimal Scheduling in the Multiserver-job Model under Heavy Traffic.''
     \textit{Proceedings of the ACM Measurement and Analysis of Computer Systems -- SIGMETRICS}:
     Volume 6, Number 3, Article 51 (Sigmetrics 2023),
     pp. 51:1 -- 51:32, December 2022.

    Isaac Grosof, Mor Harchol-Balter. ``ServerFilling: A better approach to packing multiserver jobs.''
    \textit{Proceedings of the 5th workshop on Advanced tools, programming languages, and platforms for implementing and evaluating algorithms for distributed systems (ApPLIED 2023).}
    In conjunction with PODC 2023 Conference. Article No. 7, pp. 1--5, Orlando, CA, June 2023.

    Isaac Grosof, Mor Harchol-Balter, Alan Scheller-Wolf. ``New Stability Results for Multiserver-job Models via Product-form Saturated Systems.'' \textit{Workshop on Mathematical Performance Modeling and Analysis (MAMA 2023)}. Orlando, FL, June 2023.

    Isaac Grosof, Mor Harchol-Balter, Alan Scheller-Wolf.
    ``WCFS: A new framework for analyzing multiserver systems.''
    \textit{Queueing Systems}, July 2022.

    Isaac Grosof, Naifeng Zhang, Marijn Heule.
    ``Towards the shortest DRAT proof of the Pigeonhole Principle.''
    \textit{Pragmatics of SAT.} In conjunction with SAT 2022. August 2022.

    Ziv Scully, Isaac Grosof, Michael Mitzenmacher.
    ``Uniform Bounds for Scheduling with Job Size Estimates.'' 
    \textit{13th Innovations in Theoretical Computer Science Conference (ITCS 2022)}:
    Volume 215, Article 114 (2022), pp. 114:1 -- 114:30, January 2022.
    \textbf{Invited Paper at ACM STOC 2022: Algorithms with Predictions}

    Isaac Grosof, Kunhe Yang, Ziv Scully, Mor Harchol-Balter.
    ``Nudge: Stochastically improving upon FCFS.''
    \textit{Proceedings of the ACM Measurement and Analysis of Computer Systems -- SIGMETRICS}: Volume 5, Number 2, Article 99 (2021), pp. 99:1 -- 99:25, June 2021, Beijing, China.\\
    \textbf{Winner of ACM SIGMETRICS 2021 Best Paper Award}

    Ziv Scully, Isaac Grosof and Mor Harchol-Balter.
    ``The Gittins Policy is Nearly Optimal in the M/G/k under Extremely General Conditions."
    \textit{Proceedings of the ACM Measurement and Analysis of Computer Systems -- SIGMETRICS}: Volume 4, Number 3, Article 43 (Dec 2020), pp. 43:1 -- 43:29, June 2021, Beijing, China. \\
    \textbf{Winner of INFORMS 2022 George Nicholson Award}

    Ziv Scully, Isaac Grosof, Mor Harchol-Balter.
    ``Optimal Multiserver Scheduling with Unknown Job Sizes in Heavy Traffic.''
    \textit{Performance Evaluation}, vol. 145, 2021, pp. 1-31. \\
    Conference version appeared in
    \textit{38th International Symposium on Computer Performance, Modeling, Measurements, and Evaluation (Performance 2020)},
    Milan, Italy, November 2020.

	Ben Berg, Daniel Berger, Sara McAllister, Isaac Grosof, Sathya Gunasekar, Jimmy Lu, Michael Uhlar,
	Jim Carrig, Nathan Beckmann, Mor Harchol-Balter, Greg Ganger.
	``The CacheLib Caching Engine: Design and Experiences at Scale."
	14th USENIX Symposium on Operating Systems Design and Implementation (OSDI 2020),
	Banff, Canada, November 2020.

    Isaac Grosof, Ziv Scully, Mor Harchol-Balter.
    ``Load Balancing Guardrails: Keeping Your Heavy Traffic on the Road to Low Response Times.''
    \textit{Proceedings of the ACM Measurement and Analysis of Computer Systems -- SIGMETRICS}:
    Volume 3, Issue 2, Article 42 (June 2019), pp. 42:1 -- 42:31, 2019. \\
    Conference version appeared in \textit{Proceedings of ACM Sigmetrics/Performance 2019
    Conference on Measurement and Modeling of Computer Systems (SIGMETRICS 19)}, Pheonix, AZ.
    June 2019.\\
    \textbf{Winner of ACM SIGMETRICS 2019 Best Student Paper Award} \\
    \textbf{Invited mini-plenary paper at ACM STOC 2021 TheoryFest}

    Isaac Grosof, Ziv Scully, Mor Harchol-Balter.
    ``SRPT for Multiserver Systems.''
    \textit{Performance Evaluation}, vol. 127-128, Nov. 2018, pp. 154-175. \\
    Conference version appeared in \textit{36th International Symposium on Computer Performance, Modeling, Measurements, and Evaluation (Performance 2018)}, Toulouse, France, December 2018. \\
    \textbf{Winner of IFIP Performance 2018 Best Student Paper Award}.

    Erik D. Demaine, Isaac Grosof, Jayson Lynch, and Mikhail Rudoy.
    ``Computational Complexity of Motion Planning of a Robot through Simple Gadgets.''
    \textit{Ninth International Conference on Fun with Algorithms}. La Maddalena, Italy. 2018.

    Erik D. Demaine, Isaac Grosof, and Jayson Lynch.
    ``Push-Pull Block Puzzles are Hard.''
    \textit{International Conference on Algorithms and Complexity}. Athens, Greece. 2017.

    Benjamin Grosof, Janine Bloomfield, Paul Fodor, Michael Kifer, Isaac Grosof, Miguel Calejo, and Theresa Swift.
    ``Automated Decision Support for Financial Regulatory/Policy Compliance, using Textual Rulelog.'' \textit{RuleML 2015}. Berlin, Germany. 2015.

\section{\centerline{PAPERS UNDER SUBMISSION OR REVISION}}
    \vspace{4pt}
    Zhongrui Chen, Isaac Grosof, Benjamin Berg. ``Improving Multiresource Job Scheduling with Markovian Service Rate Policies''. Under submission.

    Runhan Xie, Ziv Scully, Isaac Grosof. ``Optimal Multiserver Scheduling under General Service Constraints in Heavy Traffic''. Under revision.

    Isaac Grosof, Siva Theja Maguluri, R. Srikant. ``Convergence for Natural Policy Gradient on Infinite-State Average-Reward Markov Decision Processes''. Under submission.

    Yashaswini Murthy, Isaac Grosof, Siva Theja Maguluri, R. Srikant. ``Performance of NPG in Countable State-Space Average-Cost RL''. Under submission.
\section{\centerline{TECHNICAL REPORTS \& OTHER NONREFEREED PUBLICATIONS}}
    \vspace{4pt}

    Isaac Grosof, Yige Hong, Mor Harchol-Balter. ``Analysis of Markovian Arrivals and Service with Applications to Intermittent Overload.''
    October 2024.

    Isaac Grosof. ``Optimal Scheduling in Modern Queueing Systems.''
    \textit{Thesis}. July 2023.

    Isaac Grosof. ``Open Problem - M/G/k/SRPT Under Medium Load.''
    \textit{Stochastic Systems}. Sep. 2019.
\section{\centerline{TEACHING}}
\vspace{4pt}

\textbf{Winter 2025}: Teaching IEMS 460-1, Stochastic Processes I.
First-year graduate-level course,
proof-based look at Markov chains and Poisson processes.

\textbf{Fall 2024}: Taught IEMS 315, Stochastic Models. Junior-senior level undergrad course,
covering Markov chains and Poisson processes.
Quality of instruction rated as 5.77/6.
Student review: ``Izzy is one of the best instructors I have had at Northwestern. She is extremely passionate for teaching and cares that her students have a good grasp on the content."

\textbf{Fall 2020}: TA for CMU 15-850, Advanced Algorithms (graduate level),
assisting Prof.~Anupam Gupta.

\textbf{Fall 2019}: TA for CMU 15-857, Analytical Performance Modeling \& Design of Computer Systems
(graduate level), assisting Prof.~Mor Harchol-Balter.

\textbf{Fall 2016 \& Spring 2017}: TA for MIT 6.1220 (MIT 6.046 at the time),
Design and Analysis of Algorithms  (undergraduate level),
assisting Profs.~Shafi Goldwasser, Nir Shavit, \& Vinod Vaikunatanathan in fall 2016
and Profs.~Debayan Gupta, Aleksander Madry \& Bruce Tidor in spring 2017.

\textbf{Fall 2015}: Lab assistant for MIT 6.1010 (MIT 6.S04 at the time, 6.009 between),
Fundamentals of Programming (undergraduate level),
assisting Profs.~Adam Chipala and Srini Devadas.

\section{\centerline{EXTERNAL SERVICE}}
    \vspace{4pt}
    Co-organizer of Young Europeans in Queueing Theory 2025 Workshop. \\
    Co-publications chair for ACM SIGMETRICS 2025. \\
    TPC member for ACM SIGMETRICS/IFIP Performance 2025. \\
    TPC member for ACM SIGMETRICS/IFIP Performance 2024. \\
    Reviewed papers for ACM SIGMETRICS, IFIP Performance, Management Science,
    and IEEE/ACM Transactions on Networking, ITCS, INFORMS Journal on Computing,
    ACM Performance Evaluation Review, Stochastic Systems, Operations Research.

\section{\centerline{LEADERSHIP AND UNIVERSITY SERVICE}}
\vspace{4pt}

    \textbf{2019 - 2023}: President of the CMU Humanist League,
    a philosophical discussion group devoted to the ideal of discourse over dogma.

    \textbf{2022 - 2023}: Founding member of CSD PhD Student Council,
    a student group which organizes social bonding events within the department.
    
    \textbf{2021 - 2023}: Organizer of CSD weekly board games \& socializing dinner event.

    \textbf{2021}: Organizer of Computer Science Department (CSD) Introductory Course,
    which introduces new PhD students to the department.

    \textbf{2020 - 2021}: CSD Admissions Committee member for the Theory group, reviewed $\sim 200$ applications.

    \textbf{Fall 2020}: Organizer of CSD Theory Lunch Seminar series.

    \textbf{2020}: Led events for CSD Open House.

    \textbf{2016}: President of Random Hall, an MIT dorm of 93 residents.

    
    
\section{\centerline{EMPLOYMENT}}
    \vspace{4pt}
\textbf{Summer 2019}: Facebook, Menlo Park, CA. \\
- Research Intern to develop a machine-learning-based SSD admission policy for Facebook's
Tao caching architecture.\\
- Improved the tradeoff between hit ratio and SSD write rate.

    \textbf{Summer 2018}: Microsoft Research, Seattle, WA. \\
    - Research Intern to develop novel FPGA algorithms for linear algebra
    and solving linear programs.

\textbf{Summer 2016}: Jane Street Capital, LLC, New York City, NY.\\
- Software developer for non-obtrusive data collection about in-house trading.\\
- Software developer responsible for updating trading simulation package to accommodate new trade specification format.

\textbf{Summer 2015}: Coherent Knowledge, Seattle, WA. \\
- Knowledge Engineer to build demonstrations for the financial and natural language domains using the declarative logic programming language Ergo.

\textbf{2013 - 2014}: MIT Undergraduate Research Opportunities Program, Cambridge, MA. \\
- Researcher in Complexity Theory proving computational hardness of block puzzles and related agent motion problems.

\textbf{Summer 2014}: EMC Isilon, Seattle, WA. \\
- Software developer to replace the previous ad-hoc build platform with a modern Jenkins-based build platform.


\section{\centerline{TALKS GIVEN}}
    \vspace{4pt}
    Multiserver Stochastic Scheduling
    \begin{itemize}
        \item ECE Seminar, Northwestern, October 2024.
        \item Computer Science Seminar, Northwestern, December 2024.
    \end{itemize}

    New Lower Bounds on Optimal M/G/k Scheduling
    \begin{itemize}
        \item INFORMS Annual Meeting, October 2024.
        \item MAMA Workshop, ACM SIGMETRICS, June 2024.
    \end{itemize}

    Natural Policy Gradient for Queues
    \begin{itemize}
        \item Reinforcement Learning for Stochastic Networks, June 2024.
    \end{itemize}

    Analyzing Queues with Markovian Arrivals and Markovian Service
    \begin{itemize}
        \item \textbf{Tutorial} at ACM SIGMETRICS, June 2024.
        \item TU Eindhoven Seminar, June 2024.
    \end{itemize}

    The RESET and MARC Techniques for Multiserver-job Analysis
    \begin{itemize}
        \item IFIP Performance, November 2023.
        \item INFORMS Annual Meeting, October 2023.
        \item ACM SIGMETRICS Student Research Competition, June 2023.
    \end{itemize}

    Relative Arrivals: A New Drift Method for Time-Varying Arrivals
    \begin{itemize}
        \item Northwestern University, November 2023.
        \item SNAPP online seminar, October 2023.
    \end{itemize}

    Optimal Scheduling in Multiserver Queues (Thesis)
    \begin{itemize}
        \item Georgia Tech ISyE Student Seminar, September 2023.
        \item CMU: \textbf{Thesis Defense}, July 2023.
    \end{itemize}

    Multiserver Stochastic Scheduling: Analysis and Optimality (Job talk)
    \begin{itemize}
        \item Northwestern University, January 2023.
    \end{itemize}

    Optimal Scheduling in the Multiserver-job Model
    \begin{itemize}
        \item SIGMETRICS 2023.
        \item Harvard Theory Seminar, September 2022, hosted by Prof.~Michael Mitzenmacher.
        \item MIT Algorithms \& Complexity Seminar, September 2022, hosted by Prof.~Ronitt Rubenfeld.
    \end{itemize}

    The RESET Technique for Multiserver-Job Analysis
    \begin{itemize}
        \item SIGMETRICS 2023, Student Research Competition. 
    \end{itemize}

    New Stability Results for Multiserver-job Models via Product-form Saturated Systems
    \begin{itemize}
        \item MAMA workshop at SIGMETRICS 2023.
    \end{itemize}

    Stochastic Scheduling with Predictions
    \begin{itemize}
        \item STOC 2022, invited speaker in the Algorithms with Predictions workshop.
    \end{itemize}

    Work-Conserving Finite-Skip Models
    \begin{itemize}
        \item CMU Theory Lunch Seminar, January 2022.
        \item MIT LIDS Seminar, March 2022, hosted by Prof.~Eytan Modiano.
        \item CORS/INFORMS International Meeting 2022
    \end{itemize}

    Multiserver-Job Systems
    \begin{itemize}
        \item INFORMS Annual Meeting 2021.
    \end{itemize}

    Nudge: Stochastically Improving upon FCFS
    \begin{itemize}
        \item SIGMETRICS 2021 
    \end{itemize}

    Asymptotically Optimal Multiserver Scheduling
    \begin{itemize}
        \item UW Theory Seminar, March 2021, hosted by Prof.~Anna Karlin.
        \item YEQT 2021, invited speaker.
    \end{itemize}

    Stability for Two-class Multiserver-job Systems
    \begin{itemize}
        \item CMU Theory Lunch Seminar, October 2020.
    \end{itemize}

    Load Balancing Guardrails: Keeping Your Heavy Traffic on the Road to Low Response Times
    \begin{itemize}
        \item SIGMETRICS 2019
        \item INFORMS Annual Meeting 2019
        \item Ohio State University Seminar, January 2020, hosted by Prof.~Ness Shroff
    \end{itemize}

    M/G/k/SRPT under Medium Load
    \begin{itemize}
        \item INFORMS Annual Meeting 2018, Open Problem Session.
    \end{itemize}

    SRPT for Multiserver Systems
    \begin{itemize}
        \item MAMA workshop at SIGMETRICS 2018
        \item CMU Theory Lunch Seminar, September 2018.
        \item INFORMS Annual Meeting 2018
        \item Columbia Seminar, November 2018, hosted by Prof.~Augustin Chaintreau
        \item Performance 2018
    \end{itemize}

 \section{\centerline{PROJECTS}}
    \vspace{4pt}
\textbf{2018 - present} Programmatically-generated artwork:\\
\url{https://isaacg1.github.io/2018/12/06/programmatically-generated-artwork.html}
 
\textbf{2014 - 2017}: Author of new programming language: \textit{Pyth}\\
- Pyth is one of the best programming languages for solving tasks with the shortest possible programs.\\
- Available at \url{https://github.com/isaacg1/pyth}

\iffalse
\section{\centerline{ SKILLS }}
\vspace{4pt}
Strong mathematical background, including abstract logic, probability, and mathematical proofs. \\
Proficient with Python, OCaml, Rust, C++. \\
Experience with Haskell, C, Java.\\
Experience in functional programming (e.g., Scheme) and logic programming (e.g., Prolog). \\
Experience with Linux, Linux shell scripting. \\
\fi

\end{resume}
\end{document}
